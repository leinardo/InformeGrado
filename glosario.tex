%% ---------------------------------------------------------------------------
%% paNotation.tex
%%
%% Notation
%%
%% $Id: paNotation.tex,v 1.15 2004/03/30 05:55:59 alvarado Exp $
%% ---------------------------------------------------------------------------

\cleardoublepage
\renewcommand{\nomname}{Lista de símbolos y abreviaciones}
\markboth{\nomname}{\nomname}
\renewcommand{\nompreamble}{\addcontentsline{toc}{chapter}{\nomname}%
\setlength{\nomitemsep}{-\parsep}
\setlength{\itemsep}{10ex}
}

%%
% Símbolos en la notación general
% (es posible poner la declaración en el texto
%%

% \symg[t]{$\sys{\cdot}$}{Transformación realizada por un sistema}
% \symg[yscalar]{$y$}{Escalar.}
% \symg[zconjugado]{$\conj{z}$}{Complejo conjugado de $z$}
% \symg[rcomplexreal]{$\Re(z)$ o $z_{\Re}$}{Parte real del número complejo $z$}
% \symg[icompleximag]{$\Im(z)$ o $z_{\Im}$}{Parte imaginaria del número
%                                         complejo $z$}
% \symg[jimaginario]{$j$}{$j=\sqrt{-1}$}
% \symg[xvector]{$\vct{x}$}{Vector. \newline\hspace{1mm}%
%   $\vct{x}=\left[ x_1 \; x_2 \; \ldots \; x_n \right]^T =
%   \begin{bmatrix}
%     x_1 \\ x_2 \\ \vdots \\ x_n
%   \end{bmatrix}$}

% \symg[amatrix]{$\mat{A}$}{Matriz. \newline\hspace{1mm}%
%   $\mat{A} =
%   \begin{bmatrix}
%     a_{11} & a_{12} & \cdots & a_{1m}\\
%     a_{21} & a_{22} & \cdots & a_{2m}\\
%     \vdots & \vdots & \ddots & \vdots\\
%     a_{n1} & a_{n2} & \cdots & a_{nm}\\
%   \end{bmatrix}$}

% \symg[C]{$\setC$}{Conjunto de los números complejos.}

%%
% Algunas abreviaciones
%%

\syma{DCILab}{Laboratorio de Diseño de Circuitos Integrados}
\syma{RIS}{Redes Inalámbricas de Sensores}
\syma{FPGA}{Arreglo de Compuertas Programables}
\syma{SiRPA}{Sistema de Reconocimiento de Patrones Acústicos}
\syma{HDL}{Lenguaje de Descripción de Hardware}
\syma{SoC}{Sistema integrado en un chip}
\syma{ASIC}{Circuito Integrado de Aplicación Específica}
\syma{MOSIS}{Servicios de Implementación, Metal, Óxido, Semiconductor}
\syma{EDA}{Automatización de Diseño Electrónico}
\syma{Script}{Archivo que contiene un conjunto de órdenes e instrucciones, que implementan una función de forma programática.}
\syma{ICs}{Circuitos Integrados}
\syma{CBIC}{Circuito Integrado basado en celdas estándar}
\printnomenclature[20mm]

%%% Local Variables:
%%% mode: latex
%%% TeX-master: "paMain"
%%% End:
