\chapter*{Resumen}
%\thispagestyle{empty}
En este documento se expone una metodología de diseño semipersonalizado de circuitos integrados digitales, mediante el uso de las herramientas Synopsys. El flujo de diseño se efectúa mediante una estructura de scripting, principalmente en lenguajes TCL, Bash, y PERL, congruentes con la semántica de las herramientas de Synopsys. Se expone como se desarrolla la jerarquía de archivos y directorios para ubicar los archivos fuente y los productos de la ejecución del flujo.

Se analiza la efectividad del flujo, sometiendo el RTL de un microprocesador de aplicación específica (ASP) en arquitectura RISC-V. Se expone la integración de una celda física para la unidad de punto flotante del ASP y las celdas físicas de las memorias para datos y programa del ASP. El análisis se realiza sobre los reportes generados por la herramienta, y sobre una verificación determinística sobre el RTL y los GLN, mediante simulaciones generadas a partir de bancos de prueba (testbench) derivados de un modelo dorado de alto nivel.


\bigskip

\textbf{Palabras clave:} \scriptKeywords

\clearpage
\chapter*{Abstract}
\thispagestyle{empty}

This document presents a semipersonalized design methodology of digital integrated circuits, by using of Synopsys tools. The design flow is made through a scripting structure, mainly in languages like TCL, Bash, and PERL, wich are consistent with the semantics of Synopsys tools. This work discusses how the file and directory hierarchy is developed to locate source files and output products.

The effectiveness of the flow is analyzed by subjecting the RTL of a RISC-V arquitectured Specific Application Microprocessor (ASP). It discusses the integration of a physical cell for the ASP floating-point unit and the physical cells of the ASP data and program memories. The analysis is performed on the reports generated by the tool, and deterministic verification on the RTL and GLN, through simulations generated from test benches, that derives from a golden reference model implementet in a high level programming language.

\bigskip

\textbf{Keywords:} \scriptKeywords 

\cleardoublepage

%%% Local Variables: 
%%% mode: latex
%%% TeX-master: "main"
%%% End: 
