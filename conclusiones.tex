\chapter{Conclusiones y Recomendaciones}

\section{Conclusiones}
Partiendo de una estructura básica para la integración de diseños codificados en RTL se ha logrado diseñar una jerarquía de archivos y directorios que permite implementar exitosamente un flujo de diseño de circuitos integrados digitales.

Mediante el uso de esta jerárquía fue posible implementar una unidad aritmética de punto flotante y demostrar que es correcta y funcional. Así mismo se demostró que es posible incorporar \textit{IP Cores} dentro del diseño y que los mismos son correctos y funcionales.

Debido a particularidades que se escapan de la visión de este trabajo no fue posible demostrar la funcionalidad del microprocesador RISC-V. Debido a deficiencias en la estructura de su RTL, sin embargo, quedó demostrado que el flujo es eficaz y eficiente.

Realizar una evaluación cuantitativa sobre la estructura de \textit{scripting} diseñada no es posible ya que responde a la visión de una única persona, y su paradigma particular de trabajo. Sin embargo, desde una perspectiva cualitativa y subjetiva, se considera la estructura de \textit{scripting} eficiente, pues manifiesta atributos como: regularidad, localidad y continuidad.

\section{Recomendaciones}

Se recomienda realizar una iteración en el diseño del microprocesador ASP, enfocado a incorporar una unidad para la programación de las memorias y permitir el \textit{booteo} del microprocesador.

Finalmente recomienda incorporar en una iteración futura optimizaciones asociadas con el área y la potencia de los módulos, incorporando técnicas como \textit{clock gating}, las cuales no fueron utilizadas en este proyecto.